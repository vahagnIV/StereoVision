\documentclass[a4paper,10pt]{article}
\usepackage[utf8]{inputenc}
\usepackage{amsmath, amsfonts, graphicx}
\usepackage{hyperref, indentfirst}
%opening
\title{Real Projective Space}
\author{}

%\vspace{-15ex}
\date{}


\begin{document}
\maketitle
\section{Definition of $\mathbb{RP}^n$}

Real projective space $\mathbb{RP}^n$ is the set of lines in $\mathbb{R}^{n+1}$ passing through the origin. Given a Cartesian coordinate system $(x_1,\ldots, x_{n+1})$ in $\mathbb{R}^{n+1}$ there are $n+1$ natural charts of $\mathbb{RP}^n$ defined as follows:

\begin{equation}
 \xi_i^{(k)} = \frac{x_i}{x_k}, \quad x_k\neq 0 \label{chartk},
\end{equation}

with the transition functions between the $j$-th and $k$-th charts:

\begin{equation}
 \xi^{(k)}_i = \frac{\xi^{(j)}_i }{\xi^{(j)}_k }
\end{equation}

For the most dimensions the {\it atlas } of this charts is overcomplete \cite{mhopkins}.

\section{$\mathbb{CP}^n$}

A straightforward generalization of real generalization of real projective spaces are their complex analogues. Although in contrast to the real projective spaces the complex ones do not have a nice interpretation from the point of view of cameras and photos, they possess nice mathematical properties which can be utilized for the study of our subject. 

By analogy with \eqref{chartk} we define $n+1$ charts as follows:

\begin{equation}
 \xi_i^{(k)} = \frac{z_i}{z_k}, \quad z_k\neq 0, \quad z, \xi \in \mathbb{C} \label{cchartk},
\end{equation}

Obviously, the projection $z\to \Re{(z)}$ projects $\xi$ on $\Re{(\xi)}$. Hence, we can think about the real projective space $\mathbb{RP}^n$ as about the real part projection of $\mathbb{CP}^n$. 

Since $\mathbb{CP}^n$ has more complex structure it geometry is more reach and we can consider some quantities which do not make sense of the real projective spaces at all.

\section{Homogeneous coordinates}

Two points $x$ and $\alpha x$  ($x \in \mathbb{R},\mathbb{C}$)are on the same line passing through the origin. Hence, from the perspective of the projective space, the coordinates $(x_1, \ldots, x_{n+1})$ are regarded as {\it homogeneous coordinates}, in contrast to {\it inhomogeneous coordinates $\xi$ \eqref{chartk}}. Clearly, the {\it homogeneous coordinates} are defined up to an arbitrary multiplier $\alpha \neq 0$. Throughout the post, we will use {\it homogeneous}  and {\it inhomogeneous} coordinates interchangeably.

\section{Hyperplanes as dual projective space}

Consider an $n$-dimensional hyperplane of $\mathbb{R}^{n+1}$ passing through the origin. It is defined as follows:

\begin{equation}
 \sum\limits_{i=1}^{n+1} a_i x_i \equiv a\cdot x = 0\label{hyperplane}
\end{equation}

Obviously, the parameters $a$ are defined up to a scalar multiplier, i.e. $\forall\alpha\neq 0$ the parameters $a$ and $\alpha a$ define the same hyperplane. This means, that the set of $n$-dimensional hyperplanes passing through the origin form a dual projective space $\mathbb{RP}^{n*}$ with $a$'s being its {\it homogeneous coordinates}. 

An invertable linear transformation $H: x \to H x$ with respect to \eqref{hyperplane} transforms the dual space as follows:
\begin{equation}
 x\to Hx, \quad a\to H^{-T} a.
\end{equation}

In the language of vector analysis they say that the hyperplanes transform covariantly.

{\tiny Strictly speaking, the co/contra-variant transformations are defined for special geometric objects called tensors only. However, since we restrict ourselves to considering only linear transformations of the coordinates, we can use those terms with regard to the coordinates and their duals. }

\section{Conics on $\mathbb{RP}^n$}

Let $C$ be an $n+1 \times n+1$ dimensional matrix. Then, if $C$ is not strictly positive, or strictly negative definite, then the equation
\begin{equation}
 x^T C x = 0\label{conic}
\end{equation}
defines a conic surface.


%The intersection of 2 $n$-dimensional hyperplanes is an $n-2$ dimensional plane.



\section{$\mathbb{RP}^2$}


In this letter I want to share an interesting fact about the parallel lines in photos and their intersection points.

 To simplify things, the photos will be considered taken by a pinhole camera, which is true under certain very weak conditions for almost all modern cameras. The coordinates $(\xi_1, \xi_2)$  on the photo are related to the world coordinates of the camera as follows:

\begin{equation}
\xi_1 = f\frac{x_1}{x_3}, \quad \xi_2 = f\frac{x_2}{x_3},\label{cameraprojection}
\end{equation}
where $f$ is the focal length of the camera lens.

The directions of the axes $\xi_1$ and $x_1$ coincide in the world $\mathbb{R}^3$ as well as those for the axes  $\xi_2$ and $x_2$. The axis $x_3$ is pointed from the viewer and coincides with the principal axis of the camera lens. 

The form of the projection \eqref{cameraprojection} hints that we can think about a photo as about a chart of $\mathbb{RP}^2$. Indeed, all points on a line passing through the center of the lens (the coordinate origin) are projected on a single point on the camera sensor. A line in the world $\mathbb{R}^3$ on the photo is indistinguishable from a plane passing through the lens center and containing that line.  

In the section below I will prove the following statement. Let us have $n$ pairs of parallel lines on a plane in the $\mathbb{R}^3$; we can draw them on the paper and put on a table. If we make a photo of those lines from an arbitrary perspective they will intersect. The interesting fact about  the intersection points is that they all lie on a single line!

\section{$\mathbb{RP}^2$ }
 Real projective space $\mathbb{RP}^2$ is the space of lines in $\mathbb{R}^{3}$ passing through the origin. The coordinates of $\mathbb{R}^3$ $(x,y,z)$ are homogeneous coordinates of $\mathbb{RP}^2$, i.e. a point $(x,y,z)$ and all points $s(x,y,z)$, $s\neq 0$ correspond to a point of $\mathbb{RP}^2$.  

Planes in $\mathbb{R}^3$ passing through the origin are defined as 
\begin{equation}
\sum a_i x_i \equiv a\cdot x= 0.
\end{equation}
 Obviously, the quantities $a_i$ are defined up to an arbitrary multiplier and hence, the planes form a dual $\mathbb{RP}^2$. 
 
 Consider a chart of $\mathbb{RP}^2$
 \begin{equation}
 \xi_1 = \frac{x_1}{x_3}, \quad \xi_2=\frac{x_2}{x_3}, \quad x_3\neq 0\label{chart}
 \end{equation}
 
A plane $a$ is projected on this chart as a line

\begin{equation}
a_1 \xi_1 + a_2 \xi_2 + a_3 = 0
\end{equation} 

Here and further we will refer to planes $a$ as "lines" meaning their projection on the certain chart. However, we should always keep in mind that not all planes have their projection on the chart. In particular, the plain 
\begin{equation}
n^{(\infty)} = (0,0,1), \quad \text{or} \quad x_3 = 0\label{z0}
\end{equation}
does not project on the chart \eqref{chart} and therefore cannot be referred as a "line". In the science of Computer Vision they call it a "line at infinity" (hence the superscript).

Two lines on $\mathbb{RP}^2$ $a$ and $b$ are parallel on the chart  \eqref{chart} if 
\begin{equation}
(a\times b )\cdot n^{(\infty)} = 0, \quad \label{parallel}
\end{equation}
Alternatively, the lines on $\mathbb{RP}^2$ intersect in the point 
\begin{equation}
x = a \times b\label{intersectionpoint}
\end{equation}

In the language of lines and planes in $\mathbb{R}^3$ this can be interpreted as follows. Obviously, any two planes passing through the origin intersect. The intersection of the planes is the line \eqref{intersectionpoint} passing through the origin. If the intersection line lies in the plane \eqref{z0} then the projections of the planes on the chart \eqref{chart} are parallel.

The same result in the language of lines on the charts of $\mathbb{RP}^2$ is interpreted as  follows. The invariant form of \eqref{intersectionpoint} shows that, generally speaking, all lines on $\mathbb{RP}^2$ intersect, however if the condition \eqref{parallel} holds, then the intersection point is not  present on the considered chart.

Now let us consider the most generic invertable transformations 
\begin{equation}
x' = H x, \quad \det H \neq 0. 
\end{equation}

Obviously a plane $a$ under this transformation  is "moving in the opposite direction"
\begin{equation}
a'=H^{-T}a\label{cotransform}
\end{equation}

These kind of transformations are called homographies and clearly form a group. 

Now, let two lines $a$ and $b$ be parallel, i.e. satisfy \eqref{parallel},  and let $H$ be a homography such that

\begin{equation}
H^{-T} \cdot a,b,n^{(\infty)}  \neq \alpha n^{(\infty)} , \quad \forall \alpha \in \mathbb{R}
\end{equation}

 It is clear that the condition $\eqref{parallel}$ is not invariant under the transformation \eqref{cotransform}. Namely,

\begin{equation}
(a'\times b')\cdot  n^{(\infty)} \neq 0
\end{equation} 
 and, hence, the transformed lines $a'$ and $b'$ intersect in the point (see\eqref{intersectionpoint}):

\begin{equation}
x' = a' \times b'
\end{equation}

It is easy to show that 
\begin{equation}
(H^{-T}n^{(\infty)}) \cdot x'  = 0
\end{equation}
and, therefore, the point of $\mathbb{RP}^2$ $x'$ on the chart \eqref{chart} lies on the line $H^{-T}n^{(\infty)}$. 

Note, that while the plane $n^{(\infty)}$ does not have a projection on the chart \eqref{chart} the projection of the  transformed line $H^{-T}n^{(\infty)}$ is well defined. In other words, a generic homography moves the line at infinity.


\section{Result}

The result is illustrated in the Fig \ref{fig:drawed}

\begin{figure}[h]
\centering
 \includegraphics[width=0.6\textwidth]{../../images/parallel.jpg}
 \caption{Parallel lines taken from an arbitrary perspective. }
 \label{fig:3dcart}
\end{figure}

\begin{figure}[h]
\centering
 \includegraphics[width=0.9\textwidth]{../../images/parallel.png}
 \caption{The parallel lines intersect on the line at infinity. }
 \label{fig:drawed}
\end{figure}

\begin{thebibliography}{1}
\bibitem{mhopkins} Hopkins M.J. (1989) Minimal atlases of real projective spaces. In: Carlsson G., Cohen R., Miller H., Ravenel D. (eds) Algebraic Topology. Lecture Notes in Mathematics, vol 1370. Springer, Berlin, Heidelberg
\end{thebibliography}


\end{document}


